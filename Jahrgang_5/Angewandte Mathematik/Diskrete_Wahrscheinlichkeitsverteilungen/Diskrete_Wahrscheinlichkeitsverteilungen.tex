% Copyright (C)  2015 Richard Bäck.
% Permission is granted to copy, distribute and/or modify this document
% under the terms of the GNU Free Documentation License, Version 1.3 or
% any later version published by the Free Software Foundation; with no
% Invariant Sections, no Front-Cover Texts, and no Back-Cover Texts.  A
% copy of the license is included in the section entitled "GNU Free
% Documentation License".

\documentclass[a4paper,10pt]{article}
\usepackage[utf8]{inputenc} % enable umlauts in the source files
\usepackage{palatino}
\usepackage{graphicx} % use of graphics
\usepackage{amsmath} % for equotation*
\usepackage{eurosym} % the euro symbol
\usepackage{fixltx2e} % super-/subscript
\usepackage{listings} % code listings
\usepackage{fancyhdr} % headers
\usepackage{hyperref} % to create references to chapters etc.
\usepackage[version=3]{mhchem} % chemical formulas
\usepackage[german]{babel} % use german headings
\usepackage[margin=1.5cm,vmargin={0pt,1cm}]{geometry}

\DeclareGraphicsExtensions{.pdf}

\setlength{\headheight}{2.5cm}
\setlength{\headsep}{0.5cm}
\setlength{\textheight}{26cm}

\pagestyle{fancy}
\lhead{Richard Bäck}
\chead{}
\rhead{\today}
\lfoot{}
\cfoot{}
\rfoot{Seite \thepage}
\renewcommand{\headrulewidth}{0.4pt}
\renewcommand{\footrulewidth}{0.4pt}

\title{Zusammenschrift zu den diskreten Wahrscheinlichkeitsverteilungen}

\begin{document}
\maketitle
\thispagestyle{fancy}

\section{Hypergeometrische Verteilung}
\subsection{Defintion}
Bei der hypergeometrische Verteilung existiert eine Grundgesamtheit
*N* und eine Teilmenge *M* mit einem besonderen Merkmal. Durch eine
Stichprobe vom Umfang *n* entsteht eine Schnittmenge von *x*
Elementen. Diese *x* Elemente sind Teil der Menge *M* und *n*. Mit der
hypergeometrischen Verteilung wird die Wahrscheinlichkeit berechnet
eben *x* Elemente bei einer Stichprobe von *n* Elementen zu ziehen.
\\
Eine hypergeometrische Verteilung identifiziert sich durch folgende
Aussagen:
\begin{itemize}
\item Die Wahrscheinlichkeit ein besonderes/gesuchtes Element zu
  ziehen verändert sich nach jedem Zug.
\item Eine Ziehung kann nur zwei Ausgänge besitzen (entweder wurde ein
  besonderes/gesuchtes Element oder ein gewöhnliches Element gezogen).
\end{itemize}

Zusammenfassung der Variablen:
\begin{description}
\item[N] Grundgesamtheit
\item[M] Besondere/gesuchte Elemente, die eine Teilmenge der
  Grundgesamtheit sind.
\item[n] Stichprobenumfang
\item[x] Anzahl der Elemente in der Stichprobe, die besonders sind.
\end{description}

\subsection{Wahrscheinlichkeiten}
\subsubsection{Bestimmte Wahrscheinlichkeit}
\begin{equation}
  \label{eq:1}
  dhypergeom(x, M, N - M, n) = \frac{\binom{M}{x} \cdot \binom{N - M}{n - x}}{\binom{N}{n}} = \text{Wahrscheinlichkeit x Elemente zu erhalten}
\end{equation}

\subsubsection{Kumulative Wahrscheinlichkeit}
\begin{equation}
  \label{eq:2}
  phypergeom(x, M, N - M, n) = \text{Wahrscheinlichkeit höchstens x Elemente zu erhalten}
\end{equation}

\subsubsection{Inverse kumulative Wahrscheinlichkeit}
\begin{equation}
  \label{eq:6}
  qhypergeom(p_{\text {höchstens x Elemente}}, M, N - M, n) = \text{x Elemente}
\end{equation}

\section{Binomial-Verteilung}
\subsection{Defintion}
Eine Grundgesamtheit *n* besteht aus besonderen und normalen
Elementen. Die Wahrscheinlichkeit ein besonderes Element zu ziehen
liegt bei $ \text{*p*} * 100 $ Prozent. Es wird nun ermittelt, wie
hoch die Wahrscheinlichkeit ist, dass in *n* *x* besondere Elemente
enthalten sind.
\\
Eine Binomial-Verteilung kann folgendermaßen identifiziert werden:
\begin{itemize}
\item Die Wahrscheinlicht ein besonderes/gesuchtes Element zu ziehen
  verändert sich nicht (auch nicht nach unendlich vielen Ziehungen).
\item Eine Ziehung kann nur zwei Ausgänge besitzen (entweder wurde ein
  besonderes/gesuchtes Element oder ein gewöhnliches Element gezogen).
\end{itemize}

Zusammenfassung der Variablen:
\begin{description}
\item[n] Grundgesamtheit
\item[p] Die Wahrscheinlichkeit aus n besondere/gesuchte Elemente zu
  erhalten.
\item[x] Die Anzahl von gezogen besonderen/gesuchten Elemente in n.
\end{description}

\subsection{Wahrscheinlichkeiten}
\subsubsection{Bestimmte Wahrscheinlichkeit}
\begin{equation}
  \label{eq:3}
  dbinom(x, n, p) = \text{Wahrscheinlichkeit x Elemente zu erhalten}
\end{equation}

\subsubsection{Kumulative Wahrscheinlichkeit}
\begin{equation}
  \label{eq:4}
  pbinom(x, n, p) = \text{Wahrscheinlichkeit höchstens x Elemente zu erhalten}
\end{equation}

\subsubsection{Inverse kumulative Wahrscheinlichkeit}
\begin{equation}
  \label{eq:5}
  qbinom(p_{\text{höchstens x Elemente}}, n, p) = \text{x Elemente aus n}
\end{equation}

\section{Poisson-Verteilung}
\subsection{Defintion}
Die Poisson-Verteilung ist aus der Binomial-Verteilung entstanden und
ist zu verwenden, so bald gilt $ n \rightarrow \infty $ und $ n
\rightarrow 0 $. Es wird mit ihr die Wahrscheinlichkeit gesucht, dass
eine bestimmte Anzahl von Ereignissen pro Einheit auftritt, wenn
bereits ein Erwartungswert für das Auftreten von Ereignissen ermittelt
wurde.
\\
Identifikation einer Poisson-Verteilung:
\begin{itemize}
\item Die Wahrscheinlicht ein besonderes/gesuchtes Element zu ziehen
  verändert sich nicht (auch nicht nach unendlich vielen Ziehungen).
\item Ein Erwartungswert für den Auftritt von Ereignissen pro Einheit
  ist ermitteltbar.
\end{itemize}

Zusammenfassung der Variablen:
\begin{description}
\item[$ \mu $] mittelere Anzahl von Auftritten eines Ereignisses pro
  Einheit
\item[x] Auftreten von einer bestimmten Anzahl von Ereignissen pro
  Einheit
\end{description}

\subsection{Wahrscheinlichkeiten}
\subsubsection{Bestimmte Wahrscheinlichkeit}
\begin{equation}
  \label{eq:7}
  dpois(x, \mu) = \text{Wahrscheinlichkeit x Ereignisse zu erhalten}
\end{equation}

\subsubsection{Kumulative Wahrscheinlichkeit}
\begin{equation}
  \label{eq:8}
  ppois(x, \mu) = \text{Wahrscheinlichkeit höchstens x Ereignisse zu erhalten}
\end{equation}

\subsubsection{Inverse kumulative Wahrscheinlichkeit}
\begin{equation}
  \label{eq:9}
  qpois(p_{\text {höchstens x Ereignisse}}, \mu) = x Ereignisse
\end{equation}

\end{document}